\documentclass{article}
%Gummi|065|=)
\title{\textbf{DOER for SH project: Opportunistic Microblogging Network}}
\author{Neil Wells\\
\texttt{ndw@st-andrews.ac.uk}}
\date{}
\begin{document}

\maketitle

\section*{Description}
Define and create a microblogging platform using an opportunistic network. It should allow smartphone users to broadcast messages to other interested users by only passing on message to users in close proximity.

\section*{Objectives}

\subsection*{Primary Objectives}
\begin{itemize}
\item Design and implement a protocol for passing messages and necessary metadata between nodes in close proximity.
\item Create a core library to manage message storage and routing.
\item Implement a simple epidemic routing algorithm to send messages to all available nodes.
\item Design a routing algorithm using user metadata to route messages while disguising message content and metadata.
\end{itemize}

\subsection*{Secondary Objectives}
\begin{itemize}
\item Create a smartphone implementation using the core library.
\item Implement a more advanced routing algorithm.
\item Design and implement a mechanism to decide whether a node is trustworthy or not.
\item Evaluate the performance of the implemented routing algorithms.
\end{itemize}

\subsection*{Tertiary Objectives}
\begin{itemize}
\item Compare the real world vs simulated performance of the routing algorithms.
\end{itemize}

\section*{Ethics}
In order to test the real world performance of the network, some metadata will be collected on users. This may include:
\begin{itemize}
\item An anonymous user ID.
\item Anonymised 'Friends list' (or equivalent) of users.
\item Times and locations of encounters between anonymous users.
\item Metadata of messages passed during encounters, including message ID and origin ID, but NOT message contents.
\end{itemize}

\section*{Resources}
This project will require an Android based smartphone or tablet. If more are needed in later stages, they will be provided by the project supervisor.

\end{document}
